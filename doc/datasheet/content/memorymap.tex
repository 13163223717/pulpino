\chapter{Memory Map}

Figure~\ref{fig:pulpino_memory_map} shows the default memory-map of PULPino
assuming 32 kB of data and instruction RAM. This can be changed in the PULPino
top-level SystemVerilog file.

\begin{figure}[H]
  \centering
  %\includegraphics[width=0.8\textwidth]{./figures/memory_map}

  \begin{bytefield}{24}
  \begin{rightwordgroup}{Instruction Memory}
  \memsection{0x0000 0000}{0x0008 0000}{7}{32kB RAM}
  \end{rightwordgroup}\\
  \begin{rightwordgroup}{Boot ROM (r/o)}
  \memsection{0x0000 8000}{0x0008 8000}{3}{512B ROM}
  \end{rightwordgroup}\\
  \memsection{}{}{3}{} \\
  \begin{rightwordgroup}{Data Memory}
  \memsection{0x0010 0000}{0x0010 8000}{7}{32kB RAM}
  \end{rightwordgroup}\\
  \memsection{}{}{3}{} \\
  \begin{rightwordgroup}{Peripherals}
  \memsection{0x1A10 0000}{}{2}{UART} \\
  \memsection{0x1A10 1000}{}{2}{GPIO} \\
  \memsection{0x1A10 2000}{}{2}{SPI Master} \\
  \memsection{0x1A10 3000}{}{2}{Timer} \\
  \memsection{0x1A10 4000}{}{2}{Event/Interrupt Unit} \\
  \memsection{0x1A10 5000}{}{2}{I2C} \\
  \memsection{0x1A10 6000}{}{2}{FLL} \\
  \memsection{0x1A10 7000}{}{2}{SoC Control}\\
  \memsection{0x1A11 0000}{}{2}{Debug Port}
  \end{rightwordgroup}
  \\
  \end{bytefield}

  \caption{\pulpino memory map.}
  \label{fig:pulpino_memory_map}

\end{figure}

\section{Interrupt Vector Table}

The IVT of PULPino follows the definition given for the \rvcore core.

\begin{table}[H]
  \caption{Interrupt/exception offset vector table}
  \label{tab:exc_table}
  \centering\begin{tabular}{@{}ll@{}} \toprule
    \textbf{Description}\hspace*{110pt}   & \textbf{Address} \\ \midrule
    Reserved - Not used                   & \signal{0x00000000} - \signal{0x00000058} \\ \hline
    Interrupt Line 23: I2C IRQ            & \signal{0x0000005C} \\ \hline
    Interrupt Line 24: UART IRQ           & \signal{0x00000060} \\ \hline
    Interrupt Line 25: GPIO IRQ           & \signal{0x00000064} \\ \hline
    Interrupt Line 26: SPI Master 0       & \signal{0x00000068} \\ \hline
    Interrupt Line 27: SPI Master 1       & \signal{0x0000006C} \\ \hline
    Interrupt Line 28: Timer A Output Cmp & \signal{0x00000070} \\ \hline
    Interrupt Line 29: Timer A Overflow   & \signal{0x00000074} \\ \hline
    Interrupt Line 30: Timer B Output Cmp & \signal{0x00000078} \\ \hline
    Interrupt Line 31: Timer B Overflow   & \signal{0x0000007C} \\ \hline
    RESET                                 & \signal{0x00000080} \\ \hline
    Illegal Instruction Exception         & \signal{0x00000084} \\ \hline
    \texttt{ECALL} Instruction            & \signal{0x00000088} \\ \hline
    Invalid Memory Access                 & \signal{0x0000008C} \\ \bottomrule
  \end{tabular}
\end{table}
