%%%%%%%%%%%%%%%%%%%%%%%%%%%%%%%%%%%%%%%%%%%%%%%%%%%%%%%%%%%%%%%%%%%%%%%
%%%%%%%%%%%%%%%%%%%%%%%%%%%%%%%%%%%%%%%%%%%%%%%%%%%%%%%%%%%%%%%%%%%%%%%
%%%%%                                                                 %
%%%%%     abstract.tex                                                %
%%%%%                                                                 %
%%%%% Author:      Florian Zaruba                                     %
%%%%% Created:     10.12.2015                                         %
%%%%% Description: Report for PULPino and Imperio architecture        %
%%%%%                                                                 %
%%%%%%%%%%%%%%%%%%%%%%%%%%%%%%%%%%%%%%%%%%%%%%%%%%%%%%%%%%%%%%%%%%%%%%%
%%%%%%%%%%%%%%%%%%%%%%%%%%%%%%%%%%%%%%%%%%%%%%%%%%%%%%%%%%%%%%%%%%%%%%%

\chapter*{Abstract}
\pulpino is an open-source microcontroller like system, based on a small 32-bit RISC-V core that was developed at ETH Zurich. The core has an IPC close to 1, full support for the base integer instruction set (RV32I), compressed instructions (RV32C) and partial support for the multiplication instruction set extension (RV32M). It implements our non-standard extensions for hardware loops, post-incrementing load and store instructions, ALU and MAC operations. To allow embedded operating systems such as FreeRTOS to run, a subset of the privileged specification is supported. When the core is idle, the platform can be put into a low power mode, where only a simple event unit is active and wakes up the core in case an event/interrupt arrives.

The \pulpino platform is available for RTL simulation, FPGA and as an ASIC in UMC 65nm (Imperio). It has full debug support on all targets. In addition we support extended profiling with source code annotated execution times through KCacheGrind in RTL simulations.

\pulpino is based on IP blocks from the PULP project, the Parallel Ultra-Low-Power Processor that is developed as a collaboration between multiple universities in Europe, including the Swiss Federal Institute of Technology Zurich (ETHZ), University of Bologna, Politecnico di Milano, Swiss Federal Institute of Technology Lausanne (EPFL) and the Laboratory for Electronics and Information Technology of Atomic Energy and Alternative Energies Commission (CEA-LETI).
