%%%%%%%%%%%%%%%%%%%%%%%%%%%%%%%%%%%%%%%%%%%%%%%%%%%%%%%%%%%%%%%%%%%%%%%
%%%%%%%%%%%%%%%%%%%%%%%%%%%%%%%%%%%%%%%%%%%%%%%%%%%%%%%%%%%%%%%%%%%%%%%
%%%%%                                                                 %
%%%%%     <file_name>.tex                                             %
%%%%%                                                                 %
%%%%% Author:      <author>                                           %
%%%%% Created:     <date>                                             %
%%%%% Description: <description>                                      %
%%%%%                                                                 %
%%%%%%%%%%%%%%%%%%%%%%%%%%%%%%%%%%%%%%%%%%%%%%%%%%%%%%%%%%%%%%%%%%%%%%%
%%%%%%%%%%%%%%%%%%%%%%%%%%%%%%%%%%%%%%%%%%%%%%%%%%%%%%%%%%%%%%%%%%%%%%%

\chapter{Introduction}
% Give an overview of the problem, and put your work into a bigger
% context. Motivate the questions addressed in this work and summarize
% your contributions. Related work should also be mentioned here,
% especially if you do not have a separate chapter for it.
\pulpino is a 32-bit micro-controller like system based on IPs mostly taken from its bigger brother the \gls{pulp} project.

This project origins from the idea to open-source the PULP project. Since \gls{pulp} is a huge project \pulpino is
the first effort in doing so. The direct relation to the \gls{pulp} project is even expressed in the name chosen for the project: In Italian, adding an "-ino" at the end of a word usually means that word corresponds to a minimized version. It has been always one of the projects main aims to provide a simple and easy to use computing platform with extensive possibilities to communicate with the outside world.

Apart from the open source release, having a smaller platform has some tremendous advantages for the \gls{pulp} project as well. The \pulpino platform easily allows us to evaluate new features without considering the overhead of the whole \gls{pulp}
platform in the first place. This is true regarding simple RTL simulation as well as for Synthesis estimates.

In addition to the opportunity stated above there is still the educational aspect of the project. Due to its simplicity it can be of great value for students who want to gather deep understanding of the basic building blocks of a micro-controller like system. This relates to the SoC architecture as well as the idea and construction of a RISC core. It is often useful for ones understanding of a concept to have the possibility to observe a working implementation.

Last but not least, we hope that the open sourcing of \pulpino helps the open source hardware community to gather more momentum in the development of software tools necessary to simulate, synthesize and manufacturing open ASIC designs.


\section{General Overview}

\pulpino provides you with a 32-bit Harvard architecture (e.g. it has physically and logically separated instruction and data RAMs). At its heart it has a \gls{RISC} core operating. We currently support two different instruction sets (ISA) for two distinct cores. This can either be our OpenRISC core OR1ON or our RISC-V implementation RI5CY. The cores are pin compatible and can therefore be swapped at one's convenience. Both cores process with a four stage in-order pipeline.

The core has debug support enabled through the Advanced Debug Unit (ADB) partially adapted from the OpenCores project. The debug unit provides outside world communication via a standardized JTAG TAP. The core region (including the core, the debug unit and the RAMs)
is communicating over a standard \gls{AXI}.
A dedicated AXI to APB bridge connects the internal AXI bus to the (slower) \gls{APB}. Both bus specifications are part of the\gls{AMBA} specification.

\section{Document Structure}

This document is separated into two different parts. Part I deals with the general concept behind \pulpino and everything that is needed to start developing programs and/or specialized IPs easily. It starts with an introduction to the build framework and the project structure so that the reader can easily follow along. If you want to have the sources while reading the chapters I would highly recommend to start with reading the appendix on how to check out the project and get everything up and running.

I then aim to give a more detailed description of the overall architecture, the different IP cores and their peculiarities.This section concludes in a explanation of the functional verification framework that is shipped alongside \pulpino.

The second part contains ASIC (Imperio) specific information. It gives insight on the measures taken for the tape-out as well as chip related information and concludes with a chip data sheet. Since Imperio is the ASIC of \pulpino everything explained in the first part of the document is directly applicable to Imperio as well.

In the appendix you can find a summary of details needed to start developing for \pulpino. This includes a register description anda API description amongst others and it is supposed to act as a quick reference card for application developers.

\section{Related Work}

At the moment one can see a landsliding transition happening in the open-source hardware community. This began with the effort of the OpenRISC project in (?) which was the first open-source release of a micro-controller like architecture.
At the moment this is currently climaxing with Berkely's RISCV project and conferences on a regular basis that focus entirely on open source hardware development (e.g.: ORCONF and RISCV Workshops).

Although there are several implementations of openRISC and RISC-V cores freely available the PULP project group has always been one of the early adaptors of both ISAs, hence we developed and maintain our own cores from the very beginning. This has the advantage of being leading edge from the very beginning and being able to improve our cores in terms of area and power.One principle idea of \pulpino is to give a well established and silicon proven design back to the community on which they can build their implementations and/or extensions on.

\subsection{OpenRISC}

OpenRISC is a project started by the OpenCores community in order to establish an open source ISA and provide a reference implementation in Verilog. The first core design was called OpenRISC 1200 (OR1200) and has been published under GNU General Public License (GNU GPL). Based on that core design they have implemented a SoC variant called ORPSoC (OpenRISC Reference Platform System-on-Chip). Unfortunately until the time of this writing there has been no open source ASIC implementation of ORPSoC but the project is still under maintenance and can be run on FPGAs as well.

Nevertheless there are numerous commercial ASIC implementations based on OpenRISC 1200 and ORPSoC, for example BA12 from Beyond Semiconductor or SD1106 E by Samsung used as a main processing unit in there \gls{DTV} devices.

\subsection{RISC-V}

RISC-V is a project initiated by the electrical engineering department of University of California Berkely (UCB). It aims to create an open and freely available Instruction Set Architecture (ISA) standard. The design of the ISA aims satisfy a very broad field of application purposes. Ranging from small scale micro-controllers to full-blown out-of-order many core architectures.

Specifically interesting in relation to the present work is their Z-scale implementation. It features a 32-bit 3-stage single-issue in-order pipeline with support for the RV32IM ISA (integer base arithmetics and multiplication) and M/U privilege modes. Communication with the memories takes place over a 32bit AHB-lite bus.

Currently UCB provides to versions of the z-scale core. One is written in their own hardware description language (HDL) called chisel while the other implementation is conducted entirely in Verilog. Both are distributed under a 3-clause BSD license.

The emerging ecosystem that comes along with the growing popularity of the RISC-V ISA comes in handy for the \pulpino project as well. The various virtual platforms provided by UCB can emulate code that will natively run on \pulpino. In addition it provides us with a tool-chain that supports the official RISC-V ISA.

%TODO
\subsection{Spain University}

\subsection{lowRISC}

lowRISC is a community project that aims to create a fully open hardware system. This also includes the processor design which they intend to base on the RISC-V ISA. Their final aim is to start volume production of open silicon and open source PCB designs.

They plan to have specialized cores entirely dedicated to external I/O (so called minion cores). This will give hardware support for basic tasks such as shifting data in or out more efficiently. In a more final implementation minion cores should also be able to handle more complex communication protocols like Ethernet or USB for example. lowRISC's current idea is to use the RISC-V core developed at ETH Zurich for their minion cores implementation. The very same core is used in for \pulpino's ASIC implementation described in the second part of this document.



%%% Local Variables:
%%% mode: latex
%%% TeX-master: "../report_template"
%%% End:
