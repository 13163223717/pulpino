%%%%%%%%%%%%%%%%%%%%%%%%%%%%%%%%%%%%%%%%%%%%%%%%%%%%%%%%%%%%%%%%%%%%%%%
%%%%%%%%%%%%%%%%%%%%%%%%%%%%%%%%%%%%%%%%%%%%%%%%%%%%%%%%%%%%%%%%%%%%%%%
%%%%%                                                                 %
%%%%%     <file_name>.tex                                             %
%%%%%                                                                 %
%%%%% Author:      <author>                                           %
%%%%% Created:     <date>                                             %
%%%%% Description: <description>                                      %
%%%%%                                                                 %
%%%%%%%%%%%%%%%%%%%%%%%%%%%%%%%%%%%%%%%%%%%%%%%%%%%%%%%%%%%%%%%%%%%%%%%
%%%%%%%%%%%%%%%%%%%%%%%%%%%%%%%%%%%%%%%%%%%%%%%%%%%%%%%%%%%%%%%%%%%%%%%

\chapter{Introduction}
% Give an overview of the problem, and put your work into a bigger
% context. Motivate the questions addressed in this work and summarize
% your contributions. Related work should also be mentioned here,
% especially if you do not have a separate chapter for it.
\pulpino is a micro-controller like system based on IPs mostly taken from its bigger brother the \gls{pulp} project.

This project origins from the idea to open-source the PULP project. Since \gls{pulp} is a huge project \pulpino is
the first effort in doing so. The direct relation to the \gls{pulp} project is even expressed in the name chosen for
the project: In Italian, adding an "-ino" at the end of a word usually means that word corresponds to a minimalized version.

Apart from the open source release, having a smaller platform has some tremendous advantages for the \gls{pulp} project as
well. The \pulpino platform easily allows us to evaluate new features without considering the overhead of the whole \gls{pulp}
platform in the first place. This is true regarding simple RTL simulation as well as for Synthesis estimates.

In addition to the opportunity stated above there is still the educational aspect of the project. Due to its simplicity
it can be of great value for students who want to gather deep understanding of the basic building blocks of a
micro-controller like system. This relates to the SoC architecture as well as the idea and construction of a RISC core.
It is often useful for ones understanding of a concept to have the possibility to observe a working implementation.

\section{General Overview}

\pulpino implements a simple Harvard architecture (e.g. it has physically separated instruction and data RAMs). At its heart it
has a \gls{RISC} core operating. We currently support two different instruction sets (ISA) for two distinct cores. This can either be
our OpenRISC core OR1ON or our RISC-V implementation RI5CY. The cores are pin compatible and can therefore be swapped at one's
convenience. Both cores process with a four stage in-order pipeline.

The core has debug support enabled through the Advanced Debug Unit (ADB) partially adapted from the OpenCores project. The debug
unit provides outside world communication via standard JTAG TAP. The core region (including the core, the debug unit and the RAMs)
is communicating over a standard \gls{AXI}.

A dedicated AXI to APB bridge connects the internal AXI bus to the (slower) \gls{APB}. Both bus specifications are part of the
\gls{AMBA} specification.

\section{Document Structure}

This document is separated into two different parts. Part I deals with the general concept behind \pulpino and
everything that is needed to start developing programs and/or specialized IPs easily. It starts with an introduction to
the build framework and on how to get everything up and running so that the reader can easily follow along.

I then aim to give a more detailed description of the overall architecture, the different IP cores and their peculiarities.
This section concludes in a explanation of the functional verification framework that is shipped alongside the \pulpino project.

The second part contains ASIC (Imperio) specific information. It gives insight on the measures taken for the tape-out as
well as chip related information and concludes with a chip data sheet. Since Imperio is the ASIC of \pulpino everything explained in the
first part of the document is directly applicable to Imperio as well.

In the appendix you can find a summary of details needed to start developing for \pulpino. This includes a register description and
a API description amongst others and it is supposed to act as a quick reference card for application developer.


%%% Local Variables:
%%% mode: latex
%%% TeX-master: "../report_template"
%%% End:
