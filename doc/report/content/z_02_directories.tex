%%%%%%%%%%%%%%%%%%%%%%%%%%%%%%%%%%%%%%%%%%%%%%%%%%%%%%%%%%%%%%%%%%%%%%%
%%%%%%%%%%%%%%%%%%%%%%%%%%%%%%%%%%%%%%%%%%%%%%%%%%%%%%%%%%%%%%%%%%%%%%%
%%%%%                                                                 %
%%%%%     z_02_directories.tex                                        %
%%%%%                                                                 %
%%%%% Author:      Michael Muehlberghuber (<mbgh@iis.ee.ethz.ch>      %
%%%%% Created:     01.07.2012                                         %
%%%%% Description: A description of all files and directories         %
%%%%%              contained within this LaTeX framework.             %
%%%%%                                                                 %
%%%%%                                                                 %
%%%%% History:                                                        %
%%%%%%%%%%%%%%                                                        %
%%%%%                                                                 %
%%%%% 01-Jul-2012 (Michael Muehlberghuber - mbgh@iis.ee.ethz.ch):     %
%%%%% *) Created initial version.                                     %
%%%%%                                                                 %
%%%%%%%%%%%%%%%%%%%%%%%%%%%%%%%%%%%%%%%%%%%%%%%%%%%%%%%%%%%%%%%%%%%%%%%
%%%%%%%%%%%%%%%%%%%%%%%%%%%%%%%%%%%%%%%%%%%%%%%%%%%%%%%%%%%%%%%%%%%%%%%

\chapter{The Template Directory Structure}

This \LaTeX{} framework suitable for creating reports spreads over
various directories and files. In order to give you a short overview
of this structure, the respective directories and the contained files
are described in the following:

\begin{flushleft}
\dirtree{%
.1 /.
  .2 README \DTcomment{README file with a quick start guide.}.
  .2 Makefile \DTcomment{Makefile with some \LaTeX{} related build targets.}.
  .2 report\_template.tex \DTcomment{The main \LaTeX{} file of the report document, which further loads other (content) files.}.
  .2 bib \DTcomment{Contains bibliography related files.}.
    .3 main.bib \DTcomment{Bibliography file.}.
  .2 content \DTcomment{Contains the actual source files of your report.}.
    .3 *$.$tex \DTcomment{Here, multiple content files are provided.}.
  .2 figures \DTcomment{Contains the images which are loaded during your report.}.
    .3 eth\_logo.* \DTcomment{ETH logo in \gls{eps} and \gls{pdf} format.}.
    .3 titlepage\_logo.* \DTcomment{Titlepage logo in \gls{eps} and \gls{pdf} format.}.
    .3 asic\_pinout.* \DTcomment{Sample pinout of an \gls{asic} in \gls{eps} and \gls{pdf} format.}.
  .2 figures\_raw \DTcomment{Contains the raw sources of your figures.}.
    .3 titlepage\_logo.obj \DTcomment{Tgif titlepage logo source.}.
  .2 glossaries \DTcomment{Contains glossaries.}.
    .3 glossaries.tex \DTcomment{The glossaries file containing both the entries of the list of acronym entries and the entries of the main glossary.}.
  .2 preamble \DTcomment{Contains preamble information of the document.}.
    .3 preamble.tex \DTcomment{Preamble of the report document.}.
}
\end{flushleft}


%%% Local Variables: 
%%% mode: latex
%%% TeX-master: "../report_template"
%%% End: 
